\documentclass{article}

\usepackage{graphicx,color,tabularx,csquotes}
\graphicspath{{gfx/}}

\newcommand{\Lag}{\mathcal{L}}
\newcommand{\inkText}[2]{
  \begin{tabular}{>{\centering\hspace{0pt}}p{#1\linewidth}}
    #2
  \end{tabular}
  }

\begin{document}

\begin{figure}
  \centering
  \input{gfx/constraint.pdf_tex}
  \caption[Constraint analysis as a flowchart]{Constraint analysis as
    a flowchart. The number of degrees of freedom is abbreviated with
    \#DoF. The physical requirement that the constraints are conserved
    in time is crucial for the resulting conditions on the parameters
    $a$. The case that no primary constraints appear is trivial. Note
    that one can obtain a non-physical theory even though $\Lag$ is
    the most general Lagrangian.  Refer to
    diploma thesis for further explanation, especially
    for the option \enquote{depends on $a$}.}
\end{figure}

\end{document}